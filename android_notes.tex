\documentclass[11pt, a4paper]{book}
\usepackage{parskip}
\usepackage[top=1.5cm, left=1cm, right=1cm, bottom=2.5cm]{geometry}
\begin{document}
\chapter{Scroller}
\section{Class Overview}
This class encapsulates scrolling. The duration of the scroll can be passed
 in the constructor and specifies the maximum time that the scrolling animation
should take. Past this time, the scrolling is automatically moved to its final
stage and \verb|computeScrollOffset()| will always return false to indicate taht
scrolling is over.

\section{AppWdiget}
\section{AppWidgetManager}
Updates AppWidgetState, gets information about installed AppWidgetProviders and
other Widget related state.
\section{AppWidgetHost}
Provides the \emph{interaction} with the AppWidget source for apps, like the
home screen, that want to embed AppWidget in their UI.

You need to privde a hostid at construction, the hostId is a number of your 
choosing that should be internally unique to your app (that is, you don't 
need to worry about collisions with other apps on the system).  It's designed 
for cases where you want two unique AppWidgetHosts inside of the same 
application, so the system can optimize and only send updates to actively 
listening hosts. 


\section{AppWidgetHostView}
Provides teh glue to show AppWidgetViews. Offers automatic animation between
updates, and will try recycling old views for each incoming.

\chapter{Content Providers}
\section{Overview}
Content providers manage access to a \emph{structured set of data}. They
encapsulate the data, and provide mechanisms for defining data security. Content
providers are the standard interface that \emph{connets data in one process with
code running in another process.}

Use \verb|ContentResolver| Object in application's Context to communicate with
the provider as a client. 

\verb|ContentResolver| object communicates with the provider object, an instance
of a class that implements \emph{ContentProvider}. The provider object receives
data requests from clients. 

A content provider manages access to a central repository of data. A provider is
part of an Android application, which ofen privides its own UI for working with
the data. 

However, content providers are primarily intended to be \emph{used by other
applications}, which access teh provider using a provider client object. 

\section{Content URIs}
A content URI is a URI that indentifies data in a provider. Content URIs include
the symbolic name of teh entire provider (its authority) and name that points to
a table (a path).

\section{ContentResolver}
\begin{description}
\item[notifyChange] Notify registered obsevers that a row was updated. 
\end{description}
\end{document
